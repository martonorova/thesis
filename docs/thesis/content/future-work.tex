\chapter{Future work}

In this chapter, some possibilities for future development are discussed concerning the implemented components of this thesis project.

One possible development of the Gateway component could be to integrate it with the SimpleJSON data source plugin in order to create one tool to connect RapidMiner Server with Grafana. This would introduce several advantages compared to the current architectural setup.

Merging the Gateway component with SimpleJSON would mean, that the data format translation between these units could happen directly in Grafana. Thus, the communication overhead of the system could be reduced, as there would not be an extra component on the path of the data to be collected.


% https://grafana.com/grafana/plugins?orderBy=weight&direction=asc

Another advantage of having one component instead of two is in regard of packaging and distribution. As it was already stated, Grafana is an open source project, meaning it heavily relies on the contribution of the community. Grafana allows the development of custom Data Sources - and Panels also - that can be published on the Grafana plugin marketplace. With an integrated component it would be easier to accomplish to have a publicly available data source plugin specifically for RapidMiner Server. 

\begin{center}
	\begin{itemize}
		publishing optimized dynamic barchart, extending it for histogram
	\end{itemize}
\end{center}