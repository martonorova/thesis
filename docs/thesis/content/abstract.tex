\pagenumbering{roman}
\setcounter{page}{1}

\selecthungarian

%----------------------------------------------------------------------------
% Abstract in Hungarian
%----------------------------------------------------------------------------
\chapter*{Kivonat}\addcontentsline{toc}{chapter}{Kivonat}

Manapság egyre nagyobb hangsúlyt kapnak azok a technológiák, amelyek az adatok tárolásának, feldolgozásának és bemutatásának a problémájával foglalkoznak. Az iparnak szembe kell néznie a sokszínű adarforrásokból származó információk integrálásának kihívásokkal teli feladatával. Ugyan léteznek már erre megoldást kínáló ipari szabványok, azonban ezek átjárhatósága limitált.

Ezen szakdolgozat keretein belül ismertetem a leginkább elterjedt adatforrások jellemzőit és felhasználási területeit. Bemutatom az egyik legelterjedtebb, nyílt forrású adatvizualizációs platformot - a Grafanát - különös figyelmet fordítva a felhasználói interaktivitást lehetővé tevő funkcióira. Ismertetem a Grafana és két adatforrás (RapidMiner Server és egy Python alapú implementáció) közti lehetséges integráció tervezési és megvalósítási szempontjait, lépéseit. A létrejött architektúra lehetővé teszi az adatforrások és a Grafana közti kétirányú kommunikációt. Kiegészítve a Grafana beépített bővítményeit további funkcionalitást vezettem be annak érdekében, hogy a felhasználók könnyeben tudjanak hatékonyan lekérdezéseket küldeni az adatforrások felé, illetve hogy egyszerűbben kaphassanak egy konzisztens képet a megjelenő adatokról. A megvalósítás részletezését követően kielemzem a RapidMiner Server és a Grafana közti átjáró komponens előnyeit és hátrányait, valamint javaslatot teszek a jövőbeni esetleges fejlesztésekre is. Végül röviden megemlítem a szakdolgozat témájához hasonló feladatokat megoldani kívánó munkákat.




\vfill
\selectenglish


%----------------------------------------------------------------------------
% Abstract in English
%----------------------------------------------------------------------------
\chapter*{Abstract}\addcontentsline{toc}{chapter}{Abstract}

This document is a \LaTeX-based skeleton for BSc/MSc~theses of students at the Electrical Engineering and Informatics Faculty, Budapest University of Technology and Economics. The usage of this skeleton is optional. It has been tested with the \emph{TeXLive} \TeX~implementation, and it requires the PDF-\LaTeX~compiler.


\vfill
\selectthesislanguage

\newcounter{romanPage}
\setcounter{romanPage}{\value{page}}
\stepcounter{romanPage}