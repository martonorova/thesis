\pagenumbering{roman}
\setcounter{page}{1}

\selecthungarian

%----------------------------------------------------------------------------
% Abstract in Hungarian
%----------------------------------------------------------------------------
\chapter*{Kivonat}\addcontentsline{toc}{chapter}{Kivonat}

Manapság egyre nagyobb hangsúlyt kapnak azok a technológiák, amelyek az adatok tárolásának, feldolgozásának és bemutatásának problémájával foglalkoznak. Az iparnak szembe kell néznie a sokszínű adarforrásokból származó információk integrálásának kihívásokkal teli feladatával. Ugyan léteznek már erre megoldást kínáló különféle szabványok, azonban ezek átjárhatósága limitált.

Ezen szakdolgozat keretein belül ismertetem a leginkább elterjedt adatforrások jellemzőit és felhasználási területeit. Bemutatom az egyik legelterjedtebb, nyílt forrású adatvizualizációs platformot - a Grafanát - különös figyelmet fordítva a felhasználói interaktivitást lehetővé tevő funkcióira. Ismertetem a Grafana és két adatforrás (RapidMiner Server és egy Python alapú implementáció) közti lehetséges integráció tervezési és megvalósítási szempontjait, lépéseit. A létrejött architektúra lehetővé teszi az adatforrások és a Grafana közti kétirányú kommunikációt. Kiegészítve a Grafana beépített bővítményeit további funkcionalitást vezettem be annak érdekében, hogy a felhasználók könnyeben tudjanak hatékonyan lekérdezéseket küldeni az adatforrások felé, illetve hogy egyszerűbben kaphassanak egy konzisztens képet a megjelenő adatokról. A megvalósítás részletezését követően kielemzem a RapidMiner Server és a Grafana közti átjáró komponens előnyeit és hátrányait, valamint javaslatot teszek a jövőbeni esetleges fejlesztésekre is. Végül röviden megemlítem a szakdolgozat témájához hasonló feladatokat megoldani kívánó munkákat.




\vfill
\selectenglish


%----------------------------------------------------------------------------
% Abstract in English
%----------------------------------------------------------------------------
\chapter*{Abstract}\addcontentsline{toc}{chapter}{Abstract}

Today, the technologies aiming to solve the problem of storing, processing and visualizing data are getting more and more attention. The industry has to face the challenging task of integrating information originating from various data sources. There are already multiple standards attempting to solve this issue, however, using these together still presents many difficulties.

In this thesis project, I review the features of the main types of data sources and the use-cases, where they proved to be the most efficient. I introduce the de facto open source industry standard visualization tool - Grafana - focusing on its capabilities that enable user interactivity. I present the steps and considerations of designing and implementing the integration of Grafana and two data sources (RapidMiner Server and a Python data source). The created architecture grants the ability to make two-way connections between Grafana and its data sources. I also extended some built-in plugins in Grafana in order to allow users to send effective queries towards the data sources and to provide them with a more consistent visualization of the data. Following the details of the implementation, I analyze the gateway component that connects RapidMiner Server and Grafana, highlighting its advantages and drawbacks. I also make a few suggestions for future development possibilities. Finally, I briefly review the related works of this thesis project which try to solve the similar problem as the integration discussed here.


\vfill
\selectthesislanguage

\newcounter{romanPage}
\setcounter{romanPage}{\value{page}}
\stepcounter{romanPage}