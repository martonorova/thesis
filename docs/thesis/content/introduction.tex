%----------------------------------------------------------------------------
\chapter{\bevezetes}
%----------------------------------------------------------------------------


Nowadays, the question of storing, processing and displaying the data is becoming more and more important throughout every industry. The time, when the collected data was only useful for computers and specialists, passed. Today, the need for showing data in an easily understandable form is significant. It is no wonder, people through the whole hierarchy of a company would like to be well-informed about the results and the ongoing processes. In addition, it is getting highly valuable to be able to display vast amount of data in a way that even outsiders can comprehend.

Because of this trend, many technologies attempting to solve these problems have
appeared, creating a wide variety of tools which organizations can use.

In enterprise-grade environments, the use of complex systems - so called data-pipelines - are becoming increasingly common. These tools provide an integrated solution for transforming and querying data coming from data sources built with different technologies. With the help of these data-pipelines, it is possible to collect many types of data, no matter the format or the frequency.

All these things for one reason, to prepare the data for machine or human decision-making.

%---
\section{Problem definition}
%---

Every organization needs to take care of the sometimes cumbersome task of managing data. It is common, that this data does not come from one place, but from multiple sources, which can present various problems, especially if at some point, merging all the available data is necessary. It can happen for example, if the organization wants to visualization of the data to be in one place.

The most fundamental challenge is maybe the issue of the data format and the way of accessing the data. There are already numerous industrial standards available, however, using these together can cause difficulties. The problem is not only the conversion between different data formats, but also handling the different frequency of incoming information from each data source.

In addition, each of these heterogeneous components need to be configured in different ways, with varying set of parameters. Keeping up with all of the technologies, which may change over time independently from each other  can produce a heavy overhead for the data-specialists.

Thus, we can establish the notion, that a universal tool for integrating different types of data sources is highly desired to allow the visualization of the recorded information in one place.

%---
\section{Motivation}
%---

%\begin{itemize}
%	\item one visualization tool (Grafana) for multiple datasources in on place (one consistent way of visualizing data)
%	\item open-source development
%	\item integration task -> get familiar with many new technologies
%	\item Grafana: de facto open-source visualization tool
%\end{itemize}

There are numerous reasons why I choose to work on this thesis project. In this section I try to collect them together in order to better express my motivation towards this task.

It was expressed in the previous section, that it is quite valuable for an organization to be able to display its collected data in one visualization tool. Achieving this would allow an easily understandable and centralized way of presenting the data.

Thorough this project, I focus on integrating different data sources with a popular, open source visualization tool called Grafana. Working with this allows me to get familiar with the details of complex systems and their architecture. To combine the data sources with Grafana, I need to look into many different technologies, which surely widens my knowledge about various programming languages and design principles.

%---
\section{Goals} \label{goals}
%---

%\begin{itemize}
%	\item creating a data-gateway for accessing and visualizing data
%	\item using different datasources (different technologies, data formats)
%	\item connecting the gateway to Grafana (~industry standard for opensource data visualization)
%	\item presenting the main datasources and their features
%	\begin{itemize}
%		\item relational databases
%		\item time-series databases
%		\item key-value stores
%	\end{itemize}
%	\item discovering available options for interactivity in Grafana
%	\item design a data-gateway for connecting (two-way, duplex) different types of datasources to grafana
%	\item implement a POC data-gateway for connecting a Python based and a RapidMiner based datasource
%	\item present the advantages and disadvantages of the created gateway
%\end{itemize}


There are a couple objectives to achieve during this thesis project.

First of all, an investigation is needed about the mainly available data sources and their varying features. This is necessary to get into context and to establish further decisions concerning this work.

Following that, we have to conduct a research on the available interactivity capabilities of Grafana. It would be great, if throughout this project, we could utilize as many of them as possible and integrate them into our work. In addition, as Grafana is an open source project, there are probably plenty of possibilities for extending or customizing it in the context of interactivity. We should briefly look into it and make an attempt at implementing some simpler use-cases.

As it was already expressed above, softwares for integrating different types of data sources are indispensable. We need to design a tool that can handle this task in case of two data sources and connect them to Grafana, an industry standard for data visualization. More specifically, we need this component to be able to connect a RapidMiner data source and a Python data source to Grafana in way that allows a two-way link between them to enable further interactivity capabilities.

After the design of such tool, a sample implementation is needed to ensure the functionality of the architecture and demonstrate its effectiveness. Afterwards, we need to analyze the created gateway, discovering and presenting its advantages and disadvantages.







