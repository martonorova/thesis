%----------------------------------------------------------------------------
\chapter{\bevezetes}
%----------------------------------------------------------------------------

\textit{A bevezető tartalmazza a diplomaterv-kiírás elemzését, történelmi előzményeit, a feladat indokoltságát (a motiváció leírását), az eddigi megoldásokat, és ennek tükrében a hallgató megoldásának összefoglalását.
A bevezető szokás szerint a diplomaterv felépítésével záródik, azaz annak rövid leírásával, hogy melyik fejezet mivel foglalkozik.}


Nowadays, the question about storing, processing and displaying the data is becoming more and more important throughout every industry. The time, when the collected data was only useful for computers and specialist passed. Today, the need for showing the data in an easily understandable form is significant. And it is no wonder, people through the whole hierarchy of a company would like be well-informed about the results and the ongoing processes. In addition it is getting highly valuable to be able to display vast amount of data in a way, that even outsiders can comprehend.

Because of this trend, many technologies attempting to solve these problems have
appeared creating a wide variety of tools which organizations can use.

In enterprise-grade environments, the use of complex systems, so called 'data-pipelines' is becoming increasingly common. These tools provide an integrated solution for transforming and querying data coming from datasources built with different technologies. With the help of these data-pipelines, it is possible to collect many types of data, no matter the format or the frequency. All these things for one reason, to prepare the data for machine or human decision-making.

%---
\section{Problem definition}
%---

%---
\section{Motivation}
%---

%---
\section{Goals}
%---
