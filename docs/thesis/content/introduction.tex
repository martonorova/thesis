%----------------------------------------------------------------------------
\chapter{\bevezetes}
%----------------------------------------------------------------------------

\textit{A bevezető tartalmazza a diplomaterv-kiírás elemzését, történelmi előzményeit, a feladat indokoltságát (a motiváció leírását), az eddigi megoldásokat, és ennek tükrében a hallgató megoldásának összefoglalását.
A bevezető szokás szerint a diplomaterv felépítésével záródik, azaz annak rövid leírásával, hogy melyik fejezet mivel foglalkozik.}


Nowadays, the question of storing, processing and displaying the data is becoming more and more important throughout every industry. The time, when the collected data was only useful for computers and specialists, passed. Today, the need for showing the data in an easily understandable form is significant. It is no wonder, people through the whole hierarchy of a company would like to be well-informed about the results and the ongoing processes. In addition, it is getting highly valuable to be able to display vast amount of data in a way that even outsiders can comprehend.

Because of this trend, many technologies attempting to solve these problems have
appeared, creating a wide variety of tools which organizations can use.

In enterprise-grade environments, the use of complex systems - so called data-pipelines - are becoming increasingly common. These tools provide an integrated solution for transforming and querying data coming from datasources built with different technologies. With the help of these data-pipelines, it is possible to collect many types of data, no matter the format or the frequency. All these things for one reason, to prepare the data for machine or human decision-making.

%---
\section{Problem definition}
%---

\begin{itemize}
	\item many types of datasources, many ways of customizing them -> integration challenges
	\item standards made by the industry (data formats, accessing data, visualization) -> permeability is not easy
\end{itemize}



%---
\section{Motivation}
%---

\begin{itemize}
	\item one visualization tool (Grafana) for multiple datasources in on place (one consistent way of visualizing data)
	\item open-source development
	\item integration task -> get familiar with many new technologies
	\item Grafana: de facto open-source visualization tool
\end{itemize}

%---
\section{Goals}
%---

\begin{itemize}
	\item creating a data-gateway for accessing and visualizing data
	\item using different datasources (different technologies, data formats)
	\item connecting the gateway to Grafana (~industry standard for opensource data visualization)
	\item presenting the main datasources and their features
	\begin{itemize}
		\item relational databases
		\item time-series databases
		\item key-value stores
	\end{itemize}
	\item discovering available options for interactivity in Grafana
	\item design a data-gateway for connecting (two-way, duplex) different types of datasources to grafana
	\item implement a POC data-gateway for connecting a Python based and a RapidMiner based datasource
	\item present the advantages and disadvantages of the created gateway
\end{itemize}

